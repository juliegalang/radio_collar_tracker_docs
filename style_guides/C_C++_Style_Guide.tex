%
%  @file C_C++_Style_Guide.tex
%
%  @author Nathan Hui, nthui@eng.ucsd.edu
% 
%  @description 
%  This is the C/C++ Style guide for the Radio Telemetry Project.
%
%  This program is free software: you can redistribute it and/or modify
%  it under the terms of the GNU General Public License as published by
%  the Free Software Foundation, either version 3 of the License, or
%  (at your option) any later version.
%
%  This program is distributed in the hope that it will be useful,
%  but WITHOUT ANY WARRANTY; without even the implied warranty of
%  MERCHANTABILITY or FITNESS FOR A PARTICULAR PURPOSE.  See the
%  GNU General Public License for more details.
%
%  You should have received a copy of the GNU General Public License
%  along with this program.  If not, see <http://www.gnu.org/licenses/>.
%
%
%  DATE      WHO DESCRIPTION
%  ----------------------------------------------------------------------------
%  06/10/20  NH  Initial commit
%

\documentclass{article}
\title{Radio Telemetry Tracker C/C++ Style Guide}
\author{Nathan Hui, nthui@eng.ucsd.edu}
\date{\today\\v1}
\usepackage{listings,fullpage,xcolor}
\lstdefinestyle{customc}{
  belowcaptionskip=1\baselineskip,
  breaklines=true,
  frame=L,
  xleftmargin=\parindent,
  language=C,
  showstringspaces=false,
  basicstyle=\footnotesize\ttfamily,
  keywordstyle=\bfseries\color{green!40!black},
  commentstyle=\itshape\color{purple!40!black},
  identifierstyle=\color{blue},
  stringstyle=\color{orange},
  numbers=left
}
\lstset{
    style=customc
}
\begin{document}
\maketitle
\section{Philosopy}
In general, code should be written in a way that clearly expresses the local intent of the code, and allows for easy readability.  The following are rules that should always be followed when writing code for the RTT project.  Cases not covered by the below rules should be written in a way that is easy to understand and clearly expresses the intent behind the code.

The single most important rule when writing code is this: check the surrounding code and try to imitate it.
\section{Code Formatting}
\begin{enumerate}
    \item Always use spaces.  Tabs should be interpreted as 4 characters.
    \item Always include spaces around binary operators (\lstinline{+}, \lstinline{-}, \lstinline{*}, \lstinline{/}, \lstinline{&}, \lstinline{|}, \lstinline{>>}, \lstinline{<<}, \lstinline{>}, \lstinline{<}, etc).
    \item Include newlines between blocks of code where appropriate.  Use functions to break up and reuse code to facilitate readability and debugging.
    \item Always place braces on the new line, not indented:

\begin{lstlisting}[style=customc]
while(true)
{
    if (true)
    {
        return;
    }
    else if (false)
    {
        return;
    }
}
\end{lstlisting}

    \item Always retain braces for one-line bodies.
    \item Always lace semicolons indicating an empty statement on their own line.

\end{enumerate}
\section{Code Namespace}
\begin{enumerate}
    \item Keep symbols as concise and informative as possible - avoid the server’s ConnectionManager object that creates a server connection object.
    \item Functions and members should be camelCase.
    \item Local variables should be camelCase.
    \item Avoid global variables.  Where unavoidable, ensure the global variables do not unnecessarily pollute the namespace (e.g. prefix with the associated module or function).
    \item Reserve the names \lstinline{i}, \lstinline{j}, \lstinline{k}, \lstinline{x}, \lstinline{y}, \lstinline{z} for array and loop indexes.  Avoid using these if possible in favor of more descriptive names (e.g. \lstinline{bufIdx}, \lstinline{arrIdx}, \lstinline{rowIdx}, \lstinline{colIdx}).
    \item Enum typedefs should be suffixed with \lstinline{_e}:
\begin{lstlisting}
typedef enum myEnum
{
    ...
} myEnum_e
\end{lstlisting}
    \item Struct typedefs should be suffixed with \lstinline{_t}:
\begin{lstlisting}
typedef struct myStruct
{
    ...
} myStruct_t
\end{lstlisting}
    \item Capitalize constants
    \item Use \lstinline{<stdint.h>} types where appropriate.
    \item Use \lstinline{uint8_t} for working with bytes.
    \item Do not use macro magic to rename functions.
    \item Header inclusion guards should be \lstinline{__MODULE_H__}.  Add a comment at the corresponding \lstinline{#endif} to clarify which \lstinline{#endif} belongs to the include guard.
\end{enumerate}

\subsection{C Specific Rules}
\begin{enumerate}
    \item Symbols in each module should be named \lstinline{module_symbol}:
\begin{lstlisting}
// in sdr_record.c
void sdr_record_InitBuffers(void);
// in localization.c
void localization_GenerateEstimate(void);
\end{lstlisting}
    \item Declare all local variables at the beginning of the scope block.
\end{enumerate}
\subsection{C++ Specific Rules}
\begin{enumerate}
    \item Classes should be CamelCase
\end{enumerate}
\section{Documentation}
\begin{enumerate}
    \item Use Doxygen style comments (see https://www.doxygen.nl/manual/docblocks.html)
    \item Functions should be documented both at definition and at declaration.
    \item Use block comments to document the code throughout functions, particularly where the code is not immediately obvious.
    \item Enums and structs should be thoroughly documented.
    \item The top of every file should have the following comment block:
\begin{lstlisting}
/**
 * @file [filename]
 *
 * @author [author], [email]
 * 
 * @description 
 * [description]
 *
 * This program is free software: you can redistribute it and/or modify
 * it under the terms of the GNU General Public License as published by
 * the Free Software Foundation, either version 3 of the License, or
 * (at your option) any later version.
 *
 * This program is distributed in the hope that it will be useful,
 * but WITHOUT ANY WARRANTY; without even the implied warranty of
 * MERCHANTABILITY or FITNESS FOR A PARTICULAR PURPOSE.  See the
 * GNU General Public License for more details.
 *
 * You should have received a copy of the GNU General Public License
 * along with this program.  If not, see <http://www.gnu.org/licenses/>.
 *
 *
 * DATE      WHO DESCRIPTION
 * ----------------------------------------------------------------------------
 */
\end{lstlisting}

    Every commit must be documented in the changelog at the top of the file.  The author and name fields should be populated with the original author of the code.  Subsequent edits should be denoted with initials in the changelog.

\end{enumerate}
\end{document}